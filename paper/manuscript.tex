% Options for packages loaded elsewhere
% Options for packages loaded elsewhere
\PassOptionsToPackage{unicode}{hyperref}
\PassOptionsToPackage{hyphens}{url}
\PassOptionsToPackage{dvipsnames,svgnames,x11names}{xcolor}
%
\documentclass[
  chinese,
]{ctexart}
\usepackage{xcolor}
\usepackage{amsmath,amssymb}
\setcounter{secnumdepth}{5}
\usepackage{iftex}
\ifPDFTeX
  \usepackage[T1]{fontenc}
  \usepackage[utf8]{inputenc}
  \usepackage{textcomp} % provide euro and other symbols
\else % if luatex or xetex
  \usepackage{unicode-math} % this also loads fontspec
  \defaultfontfeatures{Scale=MatchLowercase}
  \defaultfontfeatures[\rmfamily]{Ligatures=TeX,Scale=1}
\fi
\usepackage{lmodern}
\ifPDFTeX\else
  % xetex/luatex font selection
\fi
% Use upquote if available, for straight quotes in verbatim environments
\IfFileExists{upquote.sty}{\usepackage{upquote}}{}
\IfFileExists{microtype.sty}{% use microtype if available
  \usepackage[]{microtype}
  \UseMicrotypeSet[protrusion]{basicmath} % disable protrusion for tt fonts
}{}
\makeatletter
\@ifundefined{KOMAClassName}{% if non-KOMA class
  \IfFileExists{parskip.sty}{%
    \usepackage{parskip}
  }{% else
    \setlength{\parindent}{0pt}
    \setlength{\parskip}{6pt plus 2pt minus 1pt}}
}{% if KOMA class
  \KOMAoptions{parskip=half}}
\makeatother
% Make \paragraph and \subparagraph free-standing
\makeatletter
\ifx\paragraph\undefined\else
  \let\oldparagraph\paragraph
  \renewcommand{\paragraph}{
    \@ifstar
      \xxxParagraphStar
      \xxxParagraphNoStar
  }
  \newcommand{\xxxParagraphStar}[1]{\oldparagraph*{#1}\mbox{}}
  \newcommand{\xxxParagraphNoStar}[1]{\oldparagraph{#1}\mbox{}}
\fi
\ifx\subparagraph\undefined\else
  \let\oldsubparagraph\subparagraph
  \renewcommand{\subparagraph}{
    \@ifstar
      \xxxSubParagraphStar
      \xxxSubParagraphNoStar
  }
  \newcommand{\xxxSubParagraphStar}[1]{\oldsubparagraph*{#1}\mbox{}}
  \newcommand{\xxxSubParagraphNoStar}[1]{\oldsubparagraph{#1}\mbox{}}
\fi
\makeatother


\usepackage{longtable,booktabs,array}
\usepackage{calc} % for calculating minipage widths
% Correct order of tables after \paragraph or \subparagraph
\usepackage{etoolbox}
\makeatletter
\patchcmd\longtable{\par}{\if@noskipsec\mbox{}\fi\par}{}{}
\makeatother
% Allow footnotes in longtable head/foot
\IfFileExists{footnotehyper.sty}{\usepackage{footnotehyper}}{\usepackage{footnote}}
\makesavenoteenv{longtable}
\usepackage{graphicx}
\makeatletter
\newsavebox\pandoc@box
\newcommand*\pandocbounded[1]{% scales image to fit in text height/width
  \sbox\pandoc@box{#1}%
  \Gscale@div\@tempa{\textheight}{\dimexpr\ht\pandoc@box+\dp\pandoc@box\relax}%
  \Gscale@div\@tempb{\linewidth}{\wd\pandoc@box}%
  \ifdim\@tempb\p@<\@tempa\p@\let\@tempa\@tempb\fi% select the smaller of both
  \ifdim\@tempa\p@<\p@\scalebox{\@tempa}{\usebox\pandoc@box}%
  \else\usebox{\pandoc@box}%
  \fi%
}
% Set default figure placement to htbp
\def\fps@figure{htbp}
\makeatother



\ifLuaTeX
\usepackage[bidi=basic,provide=*]{babel}
\else
\usepackage[bidi=default,provide=*]{babel}
\fi
% get rid of language-specific shorthands (see #6817):
\let\LanguageShortHands\languageshorthands
\def\languageshorthands#1{}


\setlength{\emergencystretch}{3em} % prevent overfull lines

\providecommand{\tightlist}{%
  \setlength{\itemsep}{0pt}\setlength{\parskip}{0pt}}



 


\makeatletter
\@ifpackageloaded{caption}{}{\usepackage{caption}}
\AtBeginDocument{%
\ifdefined\contentsname
  \renewcommand*\contentsname{目录}
\else
  \newcommand\contentsname{目录}
\fi
\ifdefined\listfigurename
  \renewcommand*\listfigurename{图索引}
\else
  \newcommand\listfigurename{图索引}
\fi
\ifdefined\listtablename
  \renewcommand*\listtablename{表索引}
\else
  \newcommand\listtablename{表索引}
\fi
\ifdefined\figurename
  \renewcommand*\figurename{图}
\else
  \newcommand\figurename{图}
\fi
\ifdefined\tablename
  \renewcommand*\tablename{表}
\else
  \newcommand\tablename{表}
\fi
}
\@ifpackageloaded{float}{}{\usepackage{float}}
\floatstyle{ruled}
\@ifundefined{c@chapter}{\newfloat{codelisting}{h}{lop}}{\newfloat{codelisting}{h}{lop}[chapter]}
\floatname{codelisting}{列表}
\newcommand*\listoflistings{\listof{codelisting}{列表索引}}
\makeatother
\makeatletter
\makeatother
\makeatletter
\@ifpackageloaded{caption}{}{\usepackage{caption}}
\@ifpackageloaded{subcaption}{}{\usepackage{subcaption}}
\makeatother
\usepackage{bookmark}
\IfFileExists{xurl.sty}{\usepackage{xurl}}{} % add URL line breaks if available
\urlstyle{same}
\hypersetup{
  pdftitle={WSI2ST in Colorectal Cancer},
  pdfauthor={Duxiuju et al.},
  pdflang={zh-CN},
  colorlinks=true,
  linkcolor={blue},
  filecolor={Maroon},
  citecolor={Blue},
  urlcolor={Blue},
  pdfcreator={LaTeX via pandoc}}


\title{WSI2ST in Colorectal Cancer}
\author{Duxiuju et al.}
\date{2026-02-21}
\begin{document}
\maketitle

\renewcommand*\contentsname{目录}
{
\hypersetup{linkcolor=}
\setcounter{tocdepth}{3}
\tableofcontents
}

\section{Abstract}\label{abstract}

待补充。

\section{Introduction}\label{introduction}

待补充。

\section{Results}\label{results}

\subsection{WSI2ST在外部队列中保持稳定的基因层面一致性}\label{wsi2stux5728ux5916ux90e8ux961fux5217ux4e2dux4fddux6301ux7a33ux5b9aux7684ux57faux56e0ux5c42ux9762ux4e00ux81f4ux6027}

我们首先在外部数据集中评估预测表达与实测ST之间的基因级相关性分布。\texttt{Correlation\_external.png}
显示,整体分布明显向正相关偏移,说明模型并非仅在内部训练近邻样本中有效,而是能够在跨队列条件下保留可迁移的空间分子信号。该结果支持WSI2ST在真实应用场景中的外推能力。

\begin{figure}[H]

\centering{

\pandocbounded{\includegraphics[keepaspectratio]{../Figures/Correlations/Correlation_external.png}}

}

\caption{\label{fig-corr-external}外部队列基因级相关性分布。}

\end{figure}%

\subsection{内部逐样本分析揭示``稳定性主导、异质性可解释''}\label{ux5185ux90e8ux9010ux6837ux672cux5206ux6790ux63edux793aux7a33ux5b9aux6027ux4e3bux5bfcux5f02ux8d28ux6027ux53efux89e3ux91ca}

在内部逐样本分析中,\texttt{Correlation\_internal\_persample.png}
显示不同病例间的相关性水平存在可观但有界的波动。关键点在于,性能波动主要体现为``程度差异''而非``方向反转'':大多数样本仍维持正向一致性。这种模式符合结直肠癌组织学异质性的预期,也提示误差来源更可能与局部组织构成、切片质量和微环境复杂度相关,而不是模型机制失效。

\begin{figure}[H]

\centering{

\pandocbounded{\includegraphics[keepaspectratio]{../Figures/Correlations/Correlation_internal_persample.png}}

}

\caption{\label{fig-corr-internal-sample}内部逐样本相关性比较。}

\end{figure}%

\subsection{定量汇总表明模型优势具有跨场景一致性}\label{ux5b9aux91cfux6c47ux603bux8868ux660eux6a21ux578bux4f18ux52bfux5177ux6709ux8de8ux573aux666fux4e00ux81f4ux6027}

\texttt{14CRCSummary.png} 与 \texttt{external\_Summary.png}
从目标基因覆盖、中位/均值相关性,以及相关性阈值(≥0.20、0.30、0.40、0.50)达标比例等维度进行量化。两张汇总图在内部与外部场景下给出一致结论:领先模型不仅在中心趋势上更优,也在高阈值区间维持更高基因占比,说明其提升并非由少数``易预测基因''驱动,而是分布层面的整体改进。

\begin{figure}[H]

\centering{

\pandocbounded{\includegraphics[keepaspectratio]{../ThreeLineTable/14CRCSummary.png}}

}

\caption{\label{fig-summary-internal}内部14CRC队列模型性能汇总。}

\end{figure}%

\begin{figure}[H]

\centering{

\pandocbounded{\includegraphics[keepaspectratio]{../ThreeLineTable/external_Summary.png}}

}

\caption{\label{fig-summary-external}外部队列模型性能汇总。}

\end{figure}%

\subsection{临床构成与Top基因结果支持生物学可解释性}\label{ux4e34ux5e8aux6784ux6210ux4e0etopux57faux56e0ux7ed3ux679cux652fux6301ux751fux7269ux5b66ux53efux89e3ux91caux6027}

\texttt{clinicalCharacteristics.png}
给出了数据来源、解剖部位与spot层面测序复杂度,帮助解释跨样本差异的临床与技术背景。进一步地,\texttt{Top10Genes\_14CRC.png}
与 \texttt{Top10Genes\_External.png}
显示高一致性基因在内部和外部队列中具有延续性,提示模型恢复的并非随机表达噪声,而是与组织形态耦合的稳定分子模式。

\begin{figure}[H]

\centering{

\pandocbounded{\includegraphics[keepaspectratio]{../ThreeLineTable/clinicalCharacteristics.png}}

}

\caption{\label{fig-clinical}队列临床与样本特征。}

\end{figure}%

\begin{figure}[H]

\centering{

\pandocbounded{\includegraphics[keepaspectratio]{../ThreeLineTable/Top10Genes_14CRC.png}}

}

\caption{\label{fig-top10-internal}内部14CRC队列每样本Top10一致性基因。}

\end{figure}%

\begin{figure}[H]

\centering{

\pandocbounded{\includegraphics[keepaspectratio]{../ThreeLineTable/Top10Genes_External.png}}

}

\caption{\label{fig-top10-external}外部队列每样本Top10一致性基因。}

\end{figure}%

综合来看,WSI2ST在内部与外部验证中均实现了稳健的相关性表现和可解释的基因层面保真度,为在缺乏直接ST测量时进行空间分子推断提供了可行路径。

\section{Discussion}\label{discussion}

待补充。

\section{Methods}\label{methods}

待补充。

\section{Supplements}\label{supplements}

待补充。




\end{document}
