% Options for packages loaded elsewhere
% Options for packages loaded elsewhere
\PassOptionsToPackage{unicode}{hyperref}
\PassOptionsToPackage{hyphens}{url}
\PassOptionsToPackage{dvipsnames,svgnames,x11names}{xcolor}
%
\documentclass[
  chinese,
]{ctexart}
\usepackage{xcolor}
\usepackage{amsmath,amssymb}
\setcounter{secnumdepth}{5}
\usepackage{iftex}
\ifPDFTeX
  \usepackage[T1]{fontenc}
  \usepackage[utf8]{inputenc}
  \usepackage{textcomp} % provide euro and other symbols
\else % if luatex or xetex
  \usepackage{unicode-math} % this also loads fontspec
  \defaultfontfeatures{Scale=MatchLowercase}
  \defaultfontfeatures[\rmfamily]{Ligatures=TeX,Scale=1}
\fi
\usepackage{lmodern}
\ifPDFTeX\else
  % xetex/luatex font selection
\fi
% Use upquote if available, for straight quotes in verbatim environments
\IfFileExists{upquote.sty}{\usepackage{upquote}}{}
\IfFileExists{microtype.sty}{% use microtype if available
  \usepackage[]{microtype}
  \UseMicrotypeSet[protrusion]{basicmath} % disable protrusion for tt fonts
}{}
\makeatletter
\@ifundefined{KOMAClassName}{% if non-KOMA class
  \IfFileExists{parskip.sty}{%
    \usepackage{parskip}
  }{% else
    \setlength{\parindent}{0pt}
    \setlength{\parskip}{6pt plus 2pt minus 1pt}}
}{% if KOMA class
  \KOMAoptions{parskip=half}}
\makeatother
% Make \paragraph and \subparagraph free-standing
\makeatletter
\ifx\paragraph\undefined\else
  \let\oldparagraph\paragraph
  \renewcommand{\paragraph}{
    \@ifstar
      \xxxParagraphStar
      \xxxParagraphNoStar
  }
  \newcommand{\xxxParagraphStar}[1]{\oldparagraph*{#1}\mbox{}}
  \newcommand{\xxxParagraphNoStar}[1]{\oldparagraph{#1}\mbox{}}
\fi
\ifx\subparagraph\undefined\else
  \let\oldsubparagraph\subparagraph
  \renewcommand{\subparagraph}{
    \@ifstar
      \xxxSubParagraphStar
      \xxxSubParagraphNoStar
  }
  \newcommand{\xxxSubParagraphStar}[1]{\oldsubparagraph*{#1}\mbox{}}
  \newcommand{\xxxSubParagraphNoStar}[1]{\oldsubparagraph{#1}\mbox{}}
\fi
\makeatother


\usepackage{longtable,booktabs,array}
\usepackage{calc} % for calculating minipage widths
% Correct order of tables after \paragraph or \subparagraph
\usepackage{etoolbox}
\makeatletter
\patchcmd\longtable{\par}{\if@noskipsec\mbox{}\fi\par}{}{}
\makeatother
% Allow footnotes in longtable head/foot
\IfFileExists{footnotehyper.sty}{\usepackage{footnotehyper}}{\usepackage{footnote}}
\makesavenoteenv{longtable}
\usepackage{graphicx}
\makeatletter
\newsavebox\pandoc@box
\newcommand*\pandocbounded[1]{% scales image to fit in text height/width
  \sbox\pandoc@box{#1}%
  \Gscale@div\@tempa{\textheight}{\dimexpr\ht\pandoc@box+\dp\pandoc@box\relax}%
  \Gscale@div\@tempb{\linewidth}{\wd\pandoc@box}%
  \ifdim\@tempb\p@<\@tempa\p@\let\@tempa\@tempb\fi% select the smaller of both
  \ifdim\@tempa\p@<\p@\scalebox{\@tempa}{\usebox\pandoc@box}%
  \else\usebox{\pandoc@box}%
  \fi%
}
% Set default figure placement to htbp
\def\fps@figure{htbp}
\makeatother



\ifLuaTeX
\usepackage[bidi=basic,provide=*]{babel}
\else
\usepackage[bidi=default,provide=*]{babel}
\fi
% get rid of language-specific shorthands (see #6817):
\let\LanguageShortHands\languageshorthands
\def\languageshorthands#1{}


\setlength{\emergencystretch}{3em} % prevent overfull lines

\providecommand{\tightlist}{%
  \setlength{\itemsep}{0pt}\setlength{\parskip}{0pt}}



 


\makeatletter
\@ifpackageloaded{caption}{}{\usepackage{caption}}
\AtBeginDocument{%
\ifdefined\contentsname
  \renewcommand*\contentsname{目录}
\else
  \newcommand\contentsname{目录}
\fi
\ifdefined\listfigurename
  \renewcommand*\listfigurename{图索引}
\else
  \newcommand\listfigurename{图索引}
\fi
\ifdefined\listtablename
  \renewcommand*\listtablename{表索引}
\else
  \newcommand\listtablename{表索引}
\fi
\ifdefined\figurename
  \renewcommand*\figurename{图}
\else
  \newcommand\figurename{图}
\fi
\ifdefined\tablename
  \renewcommand*\tablename{表}
\else
  \newcommand\tablename{表}
\fi
}
\@ifpackageloaded{float}{}{\usepackage{float}}
\floatstyle{ruled}
\@ifundefined{c@chapter}{\newfloat{codelisting}{h}{lop}}{\newfloat{codelisting}{h}{lop}[chapter]}
\floatname{codelisting}{列表}
\newcommand*\listoflistings{\listof{codelisting}{列表索引}}
\makeatother
\makeatletter
\makeatother
\makeatletter
\@ifpackageloaded{caption}{}{\usepackage{caption}}
\@ifpackageloaded{subcaption}{}{\usepackage{subcaption}}
\makeatother
\usepackage{bookmark}
\IfFileExists{xurl.sty}{\usepackage{xurl}}{} % add URL line breaks if available
\urlstyle{same}
\hypersetup{
  pdftitle={WSI-Inferred Spatial Transcriptomics for Colorectal Cancer},
  pdfauthor={Duxiuju et al.},
  pdflang={zh-CN},
  colorlinks=true,
  linkcolor={blue},
  filecolor={Maroon},
  citecolor={Blue},
  urlcolor={Blue},
  pdfcreator={LaTeX via pandoc}}


\title{WSI-Inferred Spatial Transcriptomics for Colorectal Cancer}
\author{Duxiuju et al.}
\date{2026-02-21}
\begin{document}
\maketitle

\renewcommand*\contentsname{目录}
{
\hypersetup{linkcolor=}
\setcounter{tocdepth}{3}
\tableofcontents
}

\section{Abstract}\label{abstract}

Spatial transcriptomics (ST) provides an interpretable molecular readout
of tissue architecture, yet its clinical scalability is constrained by
cost and tissue requirements. We developed and benchmarked a
whole-slide-image (WSI)-to-ST inference framework across internal
leave-one-patient-out cohorts and external datasets in colorectal
cancer. Across 418 target genes, the framework achieved robust
spot-level concordance with measured ST profiles and preserved
biologically meaningful spatial gradients. Comparative evaluation
against multiple state-of-the-art baselines showed consistently stronger
correlation distributions and a higher fraction of genes above practical
concordance thresholds. These findings indicate that histology-driven
virtual ST can recover substantial transcriptomic structure from routine
pathology images and may support hypothesis generation in settings where
direct ST is unavailable.

\section{Introduction}\label{introduction}

Spatial context is central to colorectal cancer biology, where
epithelial programs, stromal remodeling, and immune exclusion co-exist
within heterogeneous tissue niches. Although ST can resolve this
architecture, widespread deployment remains limited in retrospective
cohorts and routine pathology workflows. Computational inference of ST
from H\&E WSIs offers a pragmatic alternative, but the field still
requires rigorous cross-cohort validation and transparent reporting of
per-gene and per-sample behavior.

Here, we evaluate a WSI-to-ST pipeline under internal and external
settings, emphasizing clinically relevant robustness. We focus on
gene-wise correlation structure, sample-level reproducibility, and
interpretable summaries of model behavior. The study is designed to test
not only aggregate performance but also whether inferred expression
retains tissue-context fidelity across heterogeneous specimens.

\section{Results}\label{results}

\subsection{Robust cross-cohort concordance of WSI-inferred
transcriptomes}\label{robust-cross-cohort-concordance-of-wsi-inferred-transcriptomes}

We first evaluated gene-wise concordance between inferred and measured
spatial transcriptomics in internal and external settings. In the
external cohort, the global correlation distribution remained shifted
toward positive agreement, indicating that the model generalizes beyond
the training-like internal samples.

\begin{figure}[H]

\centering{

\pandocbounded{\includegraphics[keepaspectratio]{../Figures/Correlations/Correlation_external.png}}

}

\caption{\label{fig-corr-external}External cohort gene-wise correlation
distribution.}

\end{figure}%

Within the internal validation setting, per-sample correlation profiles
showed consistent performance across patients, with expected
heterogeneity in difficulty across tissue contexts.

\begin{figure}[H]

\centering{

\pandocbounded{\includegraphics[keepaspectratio]{../Figures/Correlations/Correlations_internalCV_persample.png}}

}

\caption{\label{fig-corr-internal-sample}Internal per-sample correlation
landscape (repository file:
\texttt{Correlations\_internalCV\_persample.png}; corresponds to the
requested internal per-sample correlation panel).}

\end{figure}%

Together, these results support stable transfer of histology-derived
molecular signal and suggest that performance differences are driven
more by sample complexity than by systematic model collapse.

\subsection{Quantitative benchmarking supports model-level
separation}\label{quantitative-benchmarking-supports-model-level-separation}

To quantify practical utility, we summarized target-gene coverage and
threshold-based concordance metrics. In the internal 14-CRC setting, the
leading method shows higher central tendency and a larger fraction of
genes exceeding moderate-to-high correlation cutoffs, consistent with
distributional trends in Fig. 图~\ref{fig-corr-internal-sample}.

\begin{figure}[H]

\centering{

\pandocbounded{\includegraphics[keepaspectratio]{../ThreeLineTable/14CRCSummary.png}}

}

\caption{\label{fig-tab-14crc}Internal 14-CRC summary table of model
performance.}

\end{figure}%

External summaries recapitulated this ranking, supporting model
robustness under distribution shift and independent sample
characteristics.

\begin{figure}[H]

\centering{

\pandocbounded{\includegraphics[keepaspectratio]{../ThreeLineTable/external_Summary.png}}

}

\caption{\label{fig-tab-external}External cohort summary table of model
performance.}

\end{figure}%

\subsection{Clinical and biological context of inferred
expression}\label{clinical-and-biological-context-of-inferred-expression}

Clinical composition across cohorts (dataset source, localization, and
spot-level sequencing depth surrogates) provides context for the
observed variation in model behavior.

\begin{figure}[H]

\centering{

\pandocbounded{\includegraphics[keepaspectratio]{../ThreeLineTable/clinicalCharacteristics.png}}

}

\caption{\label{fig-clinic}Clinical characteristics across internal and
external cohorts.}

\end{figure}%

At the gene level, top-ranked concordant genes remained biologically
coherent across internal and external sets, indicating that recovered
signals are not dominated by idiosyncratic sample artifacts.

\begin{figure}[H]

\centering{

\pandocbounded{\includegraphics[keepaspectratio]{../ThreeLineTable/Top10Genes_14CRC.png}}

}

\caption{\label{fig-top10-internal}Top 10 concordant genes per sample in
internal 14-CRC cohort.}

\end{figure}%

\begin{figure}[H]

\centering{

\pandocbounded{\includegraphics[keepaspectratio]{../ThreeLineTable/Top10Genes_External.png}}

}

\caption{\label{fig-top10-external}Top 10 concordant genes per sample in
external cohorts.}

\end{figure}%

Collectively, these analyses indicate that WSI-inferred ST captures
reproducible transcriptomic structure across datasets while preserving
biologically interpretable gene-level patterns.

\section{Discussion}\label{discussion}

The present analysis demonstrates that virtual ST from WSIs can reach
reproducible concordance across both internal and external colorectal
datasets, with a clear advantage for the top-performing model in this
benchmark. Importantly, improvements were not restricted to a few marker
genes but extended to distribution-level shifts in per-gene correlations
and threshold-based quality metrics.

Several limitations remain. First, performance still varies by sample
and tissue context, suggesting unresolved domain shifts in staining,
cellular composition, or section quality. Second, correlation-based
metrics do not fully capture pathway-level conservation or downstream
clinical utility. Third, retrospective evaluation cannot substitute
prospective deployment constraints.

Future work should include prospective multi-center validation,
uncertainty-aware calibration at spot level, and integration with
pathology annotation priors to improve robustness in low-signal regions.

\section{Methods}\label{methods}

\subsection{Study Design}\label{study-design}

We analyzed internal leave-one-patient-out data and external colorectal
datasets, using measured ST as reference and model-inferred ST as
prediction.

\subsection{Computational Workflow}\label{computational-workflow}

The repository pipeline follows ordered scripts: 1.
\texttt{1\_Correlation\_0705\_Parallel.R}: computes correlation outputs.
2. \texttt{2-Prepare\_gt\_pre\_csv\_for\_newh5\_Parallel.R}: prepares
matched GT/prediction matrices. 3.
\texttt{3\_newh5\_from\_csv\_0707\_Parallel.R}: builds intermediate
\texttt{newh5} assets. 4. \texttt{4\_spe\_from\_newh5.R}: constructs
\texttt{SpatialExperiment} objects and visualization outputs. 5.
\texttt{5\_three\_line\_Table.R}: exports summary tables and top-gene
reports.

\subsection{Metrics and Reporting}\label{metrics-and-reporting}

Primary endpoint: gene-wise spot-level correlation between inferred and
measured ST. We report median/mean correlation and the proportion of
genes above thresholds (0.20, 0.30, 0.40, 0.50).

\subsection{Statistical Notes}\label{statistical-notes}

Results are descriptive and benchmark-oriented; figures summarize
distributional trends across genes and samples.

\section{Supplements}\label{supplements}

\subsection{Supplementary Figures and
Tables}\label{supplementary-figures-and-tables}

\begin{itemize}
\tightlist
\item
  Correlation summaries and per-sample views are provided in
  \texttt{Figures/Correlations/}.
\item
  Cohort-level and external summary tables are provided in
  \texttt{ThreeLineTable/}.
\end{itemize}

\subsection{Reproducibility}\label{reproducibility}

All outputs reported in this manuscript are generated from scripts in
the repository root and can be reproduced via \texttt{Rscript} execution
in sequence.




\end{document}
